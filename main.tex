\documentclass[12pt]{article}

% Essential packages for mathematics
\usepackage{amsmath}    % Advanced math environments
\usepackage{amssymb}    % Additional math symbols
\usepackage{amsthm}     % Theorem environments
\usepackage{mathtools}  % Extensions to amsmath

% Algorithm packages
\usepackage[ruled,vlined]{algorithm2e}  % Popular algorithm package
% Alternative: \usepackage{algorithm,algpseudocode} % For algorithmicx

% Other useful packages
\usepackage{graphicx}   % For including figures
\usepackage{geometry}   % Page layout
\usepackage{hyperref}   % Hyperlinks
\usepackage{cleveref}   % Smart cross-referencing
\usepackage{dirtree}

% Page setup
\geometry{margin=1in}
\setlength{\parindent}{0pt}
\setlength{\parskip}{1em}

% Theorem environments
\newtheorem{theorem}{Theorem}[section]
\newtheorem{lemma}[theorem]{Lemma}
\newtheorem{corollary}[theorem]{Corollary}
\newtheorem{proposition}[theorem]{Proposition}
\theoremstyle{definition}
\newtheorem{definition}[theorem]{Definition}
\newtheorem{example}[theorem]{Example}
\theoremstyle{remark}
\newtheorem{remark}[theorem]{Remark}

% Custom math operators
\DeclareMathOperator{\argmax}{arg\,max}
\DeclareMathOperator{\argmin}{arg\,min}
\DeclareMathOperator{\E}{\mathbb{E}}
\DeclareMathOperator{\Var}{Var}
\DeclareMathOperator{\Cov}{Cov}

% Algorithm2e settings
\SetAlgoNlRelativeSize{0}  % Line numbers same size as text
\SetAlgoNoLine             % Remove vertical lines (comment to keep them)
\DontPrintSemicolon        % Don't print semicolons at end of lines

\title{Discussion on Project}
\author{Harshit Rawat}
\date{\today}

\begin{document}

\maketitle
\section{Preliminary observations}
Our project can be understood with the following structure:
\medskip
\dirtree{%
.1 Molecule name 5(aspirin trained using 5 examples).
.2 random\_seed\_42.
.3 lr\_10(Ensemble).
.4 model\_1.
.4 model\_2.
.4 model\_3.
.4 model\_4.
.4 model\_5.
.3 lr\_20.
.4 model\_1.
.4 model\_2.
.4 \ldots .
.2 random\_seed\_52.
.1 Molecule name 10(aspirin trained using 10 examples).
}


\newpage
\section{Preliminary Model Selection Framework and Results}
\begin{algorithm}[htbp]
\caption{Median Based Algorithm for Model Selection}\label{alg:Median algorithm}
\KwIn{Set of ensembles $\mathcal{E}$, each ensemble contains 5 models}
\KwOut{Best Ensemble}

\ForEach{ensemble $E \in \mathcal{E}$}{
    \ForEach{model $m \in E$}{
        Evaluate $m$ \textbf{energy\_mae} of model on it's   \textbf{personal} validation set\;
        This will be used as \textbf{model performance metric}. \;
    }
    Compute median of performance of models in $E$\;
}
Select ensemble with best median performance\;
\end{algorithm}
\begin{table}[htbp]
\centering
\small
\caption{Energy and Force MAE (mean $\pm$ std) for different numbers of examples.}
\begin{tabular}{@{}lcccc@{}}
\hline
\textbf{num\_ex} & \textbf{NEAL Energy} & \textbf{Vanilla Energy} & \textbf{NEAL Force} & \textbf{Vanilla Force} \\
& \textbf{MAE} & \textbf{MAE} & \textbf{MAE} & \textbf{MAE} \\
\hline
5  & 343.26 $\pm$ 100.34 & \textbf{288.08 $\pm$ 51.77}  & 364.96 $\pm$ 93.60  & \textbf{313.23 $\pm$ 22.08} \\
10 & 205.71 $\pm$ 23.73  & \textbf{203.09 $\pm$ 32.34}  & \textbf{242.51 $\pm$ 20.24}  & 247.44 $\pm$ 27.42 \\
20 & \textbf{108.13 $\pm$ 15.15}  & 131.57 $\pm$ 7.77   & 175.42 $\pm$ 4.13   & \textbf{173.97 $\pm$ 4.24}  \\
40 & \textbf{75.56  $\pm$ 5.01}   & 87.26  $\pm$ 16.59  & 142.85 $\pm$ 7.09   & \textbf{138.52 $\pm$ 2.94}  \\
\hline
\end{tabular}
\\[5pt] % This adds a vertical space of 5 points after the table.
Note: Values slightly different from the ones reported in ppt as I directly took avg in pandas.
\end{table}



\end{document}