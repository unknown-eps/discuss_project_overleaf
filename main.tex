\documentclass[12pt]{article}

% Essential packages for mathematics
\usepackage{amsmath}    % Advanced math environments
\usepackage{amssymb}    % Additional math symbols
\usepackage{amsthm}     % Theorem environments
\usepackage{mathtools}  % Extensions to amsmath

% Algorithm packages
\usepackage[ruled,vlined]{algorithm2e}  % Popular algorithm package
% Alternative: \usepackage{algorithm,algpseudocode} % For algorithmicx

% Other useful packages
\usepackage{graphicx}   % For including figures
\usepackage{geometry}   % Page layout
\usepackage{hyperref}   % Hyperlinks
\usepackage{cleveref}   % Smart cross-referencing
\usepackage{dirtree}
\usepackage{float}
\usepackage{multirow}
% Page setup
\geometry{margin=1in}
\setlength{\parindent}{0pt}
\setlength{\parskip}{1em}

% Theorem environments
\newtheorem{theorem}{Theorem}[section]
\newtheorem{lemma}[theorem]{Lemma}
\newtheorem{corollary}[theorem]{Corollary}
\newtheorem{proposition}[theorem]{Proposition}
\theoremstyle{definition}
\newtheorem{definition}[theorem]{Definition}
\newtheorem{example}[theorem]{Example}
\theoremstyle{remark}
\newtheorem{remark}[theorem]{Remark}

% Custom math operators
\DeclareMathOperator{\argmax}{arg\,max}
\DeclareMathOperator{\argmin}{arg\,min}
\DeclareMathOperator{\E}{\mathbb{E}}
\DeclareMathOperator{\Var}{Var}
\DeclareMathOperator{\Cov}{Cov}

% Algorithm2e settings
\SetAlgoNlRelativeSize{0}  % Line numbers same size as text
\SetAlgoNoLine             % Remove vertical lines (comment to keep them)
\DontPrintSemicolon        % Don't print semicolons at end of lines

\title{Discussion on Project}
\author{Harshit Rawat}
\date{\today}

\begin{document}

\maketitle
\section{Preliminary observations}
Our project can be understood with the following structure:
\medskip
\dirtree{%
.1 Molecule name 5(aspirin trained using 5 examples).
.2 random\_seed\_42.
.3 lr\_10(Ensemble).
.4 model\_1.
.4 model\_2.
.4 model\_3.
.4 model\_4.
.4 model\_5.
.3 lr\_20.
.4 model\_1.
.4 model\_2.
.4 \ldots .
.2 random\_seed\_52.
.1 Molecule name 10(aspirin trained using 10 examples).
}


\newpage
\section{Preliminary Model Selection Framework and Results}
\begin{algorithm}[htbp]
\caption{Median Based Algorithm for Model Selection}\label{alg:Median algorithm}
\KwIn{Set of ensembles $\mathcal{E}$, each ensemble contains 5 models}
\KwOut{Best Ensemble}

\ForEach{ensemble $E \in \mathcal{E}$}{
    \ForEach{model $m \in E$}{
        Evaluate $m$ \textbf{energy\_mae} of model on it's   \textbf{personal} validation set\;
        This will be used as \textbf{model performance metric}. \;
    }
    Compute median of performance of models in $E$\;
}
Select ensemble with best median performance\;
\end{algorithm}
\begin{table}[htbp]
\centering
\small
\caption{Energy and Force MAE (mean $\pm$ std) for different numbers of examples.}
\begin{tabular}{@{}lcccc@{}}
\hline
\textbf{num\_ex} & \textbf{NEAL Energy} & \textbf{Vanilla Energy} & \textbf{NEAL Force} & \textbf{Vanilla Force} \\
& \textbf{MAE} & \textbf{MAE} & \textbf{MAE} & \textbf{MAE} \\
\hline
5  & 343.26 $\pm$ 100.34 & \textbf{288.08 $\pm$ 51.77}  & 364.96 $\pm$ 93.60  & \textbf{313.23 $\pm$ 22.08} \\
10 & 205.71 $\pm$ 23.73  & \textbf{203.09 $\pm$ 32.34}  & \textbf{242.51 $\pm$ 20.24}  & 247.44 $\pm$ 27.42 \\
20 & \textbf{108.13 $\pm$ 15.15}  & 131.57 $\pm$ 7.77   & 175.42 $\pm$ 4.13   & \textbf{173.97 $\pm$ 4.24}  \\
40 & \textbf{75.56  $\pm$ 5.01}   & 87.26  $\pm$ 16.59  & 142.85 $\pm$ 7.09   & \textbf{138.52 $\pm$ 2.94}  \\
\hline
\end{tabular}
\\[5pt] % This adds a vertical space of 5 points after the table.
Note: Values slightly different from the ones reported in ppt as I directly took avg in pandas.
\end{table}
\begin{figure}[H]
    \centering
    \includegraphics[width=1.0\linewidth]{plots/plots_for_median_strategy_using_only_train_energy.png}
    \caption{Median strategy for combination using energy mae as perf measure.}
    \label{fig:placeholder}
\end{figure}
\newpage
\section{Model Selection via Validation Set of 500 unseen examples}
Each ensemble was evaluated on the validation set and the model with lowest validation loss was selected for each seed. To ensure a fair comparison \textbf{vanilla} models were also evaluated across their hyperparameters on validation set.
\begin{table}[htbp]
\centering
\small
\caption{Energy and Force MAE (mean $\pm$ std) for different numbers of examples.}
\begin{tabular}{@{}lcccc@{}}
\hline
\textbf{num\_ex} & \textbf{NEAL Energy} & \textbf{Vanilla Energy} & \textbf{NEAL Force} & \textbf{Vanilla Force} \\
& \textbf{MAE} & \textbf{MAE} & \textbf{MAE} & \textbf{MAE} \\
\hline
5  & \textbf{243.72 $\pm$ 41.24} & 254.80 $\pm$ 39.75  & \textbf{281.81 $\pm$ 14.55} & 289.37 $\pm$ 16.08 \\
10 & \textbf{166.55 $\pm$ 29.73} & 178.93 $\pm$ 20.54  & 232.83 $\pm$ 19.25 & \textbf{232.06 $\pm$ 14.46} \\
20 & \textbf{99.75 $\pm$ 11.23}  & 131.58 $\pm$ 7.78   & 177.79 $\pm$ 2.83  & \textbf{173.97 $\pm$ 4.24} \\
40 & \textbf{69.80 $\pm$ 3.80}   & 87.27 $\pm$ 16.58   & 140.74 $\pm$ 6.37  & \textbf{138.52 $\pm$ 2.94} \\
\hline
\end{tabular}
\end{table}
\begin{figure}[htbp]
    \centering
    \includegraphics[width=1.0\linewidth]{plots/validation on 500 examples using enerhy mae.png}
    \caption{Using energy mae with a validation set of 500 examples.}
    \label{fig:placeholder}
\end{figure}

Repeating the experiments using only force\_mae.
\begin{table}[htbp]
\centering
\small
\caption{Energy and Force MAE (mean $\pm$ std) for different numbers of examples.}
\begin{tabular}{@{}lcccc@{}}
\hline
\textbf{num\_ex} & \textbf{NEAL Energy} & \textbf{Vanilla Energy} & \textbf{NEAL Force} & \textbf{Vanilla Force} \\
& \textbf{MAE} & \textbf{MAE} & \textbf{MAE} & \textbf{MAE} \\
\hline
5  & \textbf{252.82 $\pm$ 41.68} & 254.81 $\pm$ 39.76 & \textbf{280.94 $\pm$ 14.57} & 289.37 $\pm$ 16.08 \\
10 & \textbf{173.87 $\pm$ 34.88} & 180.64 $\pm$ 20.52 & \textbf{221.50 $\pm$ 10.83} & 225.99 $\pm$ 6.40 \\
20 & \textbf{122.49 $\pm$ 17.06} & 131.60 $\pm$ 7.77  & \textbf{172.41 $\pm$ 3.43}  & 173.97 $\pm$ 4.24 \\
40 & \textbf{81.05 $\pm$ 14.19}  & 87.26 $\pm$ 16.60  & \textbf{137.20 $\pm$ 3.06}  & 138.52 $\pm$ 2.94 \\
\hline
\end{tabular}
\end{table}


\newpage
\begin{figure}[htbp]
    \centering
    \includegraphics[width=1\linewidth]{plots/Validation using 500 examples using force mae only .png}
    \caption{Using force mae of 500 examples}
    \label{fig:placeholder}
\end{figure}

\section{Ensemble Selection Problem}
The main problem of quantifying the performance on a problem can be broken in two steps
\begin{enumerate}
    \item Identify the performance of each of the 5 models in ensemble using it's cross-validation set which it has not seen in training. 
    \item Evaluating a model on it's cross-validation set gives \textbf{2 scalars} \textbf{energy\_mae} and \textbf{force\_mae}. 
\item We aim to find the optimal scalars $\mathbf{\lambda_e}$ and $\mathbf{\lambda_f}$ such that their weighted combination
\begin{equation}
    \mathbf{\lambda_e \cdot E_{\text{mae}} + \lambda_f \cdot F_{\text{mae} }}
\end{equation}
provides the best estimate of overall model performance.
\item We will get one value for each model and the goal is to combine them. Here we have multiple choices like median, mean+std, mean-std, mean/std etc.
\end{enumerate}
\newpage
\section{Spearman Correlation}
Calculation of Spearman Correlation for a complete strategy eg ($\lambda_e=1$ $\lambda_f=0$ , median for aggregation for a fixed budget num\_eg = 10 )
\begin{enumerate}
    \item For each ensemble evaluate it on a validation set of 500 examples and \textbf{use energy mae} as the \textbf{true estimate} of the performance.
    \item \textbf{Note} that as we are focusing of energy the true estimate will never change. That is for whatever values of $\lambda_e$ and $\lambda_f$ our target remains same.
    \item For each ensemble estimate it's performance using spearman correlation.
    \item Calculate spearman correlation coefficient using these two values.
    \item \textbf{Note that we do not treat different random seeds differently.}
\end{enumerate}
\begin{figure}[htbp]
    \centering
    \includegraphics[width=0.75\linewidth]{plots/spearman coorelation using energy.png}
    \caption{Spearman Correlation using Energy only }
    \label{fig:placeholder}
\end{figure}
\newpage
\begin{figure}[htbp]
    \centering
    \includegraphics[width=0.75\linewidth]{plots/Spearman coreelation using force + energy use all seeds.png}
    \caption{Spearman correlation using energy mae + force mae.}
    \label{fig:placeholder}
\end{figure}
We observe that \textbf{mean} is generally the best and force + energy has higher correlation.
\begin{table}[htbp]
\centering
\small
\caption{Energy and Force MAE (mean $\pm$ std) force plus energy used for model selection .}
\begin{tabular}{@{}lcccc@{}}
\hline
\textbf{num\_ex} & \textbf{NEAL Energy MAE} & \textbf{Vanilla Energy MAE} & \textbf{NEAL Force MAE} & \textbf{Vanilla Force MAE} \\
\hline
5  & 263.19 $\pm$ 35.52 & \textbf{254.81 $\pm$ 39.76} & 294.66 $\pm$ 19.63 & \textbf{289.37 $\pm$ 16.08} \\
10 & \textbf{174.07 $\pm$ 26.55} & 180.64 $\pm$ 20.52 & \textbf{223.58 $\pm$ 9.99} & 225.99 $\pm$ 6.40 \\
20 & \textbf{111.29 $\pm$ 18.47} & 131.59 $\pm$ 7.77  & 178.66 $\pm$ 7.00  & \textbf{173.97 $\pm$ 4.24} \\
40 & \textbf{72.71 $\pm$ 5.87}   & 87.27 $\pm$ 16.59  & 138.86 $\pm$ 3.27  & \textbf{138.52 $\pm$ 2.94} \\
50 & \textbf{64.62 $\pm$ 5.16} & 77.38 $\pm$ 3.64     & \textbf{124.11 $\pm$ 3.79} & 124.61 $\pm$ 3.93
\\
60 & \textbf{56.7 $\pm$ 5.3} & 72.00 $\pm$ 9.3     & 118.00 $\pm$ 0.91 & \textbf{117.97 $\pm$ 3.22}\\
120 & \textbf{38.35 $\pm$ 2.64} & 54.73 $\pm$ 15.06     & 88.15 $\pm$ 1.34 & \textbf{87.86 $\pm$ 0.62}
\\
\hline
\end{tabular}
\end{table}

\newpage
\section{Using method suggested by Rushikesh}
Method: For each budget,strategy and seed calculate spearman correlation. Then avg across seeds to get the result for a strategy at this budget.

\begin{figure}[htbp]
    \centering
    \includegraphics[width=0.75\linewidth]{plots/spearman correlation avg.png}
    \caption{Spearman Correlation using energy mae for different strategies averaged across seeds.}
    \label{fig:placeholder}
\end{figure}
\newpage
\begin{figure}[H]
    \centering
    \includegraphics[width=0.75\linewidth]{plots/spearman correlation using energy mae + force mae averaged across 5 seeds.png}
    \caption{Spearman correlation using energy and force averaged across 5 seeds.}
    \label{fig:placeholder}
\end{figure}
\section{Nodal Schnet Results}
\subsection{Using only 1st interaction layer and energy latent}
\begin{table}[h!]
\centering
\begin{tabular}{c|cc|cc}
\hline
\textbf{num\_examples} & \textbf{Energy (NeAL)} & \textbf{Energy (Vanilla)} & \textbf{Force (NeAL)} & \textbf{Force (Vanilla)} \\
\hline
5  & 281.67 $\pm$ 40.90 & \textbf{281.66 $\pm$ 55.97} & \textbf{317.17 $\pm$ 29.70} & 321.44 $\pm$ 28.19 \\
10 & \textbf{188.44 $\pm$ 23.28} & 191.66 $\pm$ 9.81 & \textbf{242.95 $\pm$ 16.45} & 246.20 $\pm$ 16.29 \\
20 & \textbf{120.85 $\pm$ 13.24} & 126.96 $\pm$ 14.91 & \textbf{173.66 $\pm$ 4.84} & 174.32 $\pm$ 4.32 \\
40 & 78.27 $\pm$ 9.76 & \textbf{73.78 $\pm$ 4.20} & \textbf{135.53 $\pm$ 2.69} & 136.01 $\pm$ 1.91 \\
\hline
\end{tabular}
\caption{Using guidance from only first Layer of interactions and energy latents. Mean $\pm$ std across 5 seeds.}
\label{tab:neal_vanilla_results}
\end{table}
\newpage
\subsection{Using only 3rd interaction latent and energy layer.}
\begin{table}[htbp]
\centering
\caption{Energy and Force MAE ($\text{mean} \pm \text{std}$) for varying number of examples. The best (lowest) values for energy and force are highlighted separately.}
\begin{tabular}{c|cc|cc}
\hline
\textbf{num\_ex} & \multicolumn{2}{c|}{\textbf{Energy MAE}} & \multicolumn{2}{c}{\textbf{Force MAE}} \\
 & \textbf{Neal} & \textbf{Vanilla Nodal} & \textbf{Neal} & \textbf{Vanilla Nodal} \\
\hline
5   & \textbf{275.86 $\pm$ 27.12} & 281.65 $\pm$ 55.97 & \textbf{321.09 $\pm$ 17.85} & 321.44 $\pm$ 28.19 \\
10  & \textbf{161.26 $\pm$ 13.31} & 191.66 $\pm$ 9.80  & \textbf{232.86 $\pm$ 8.46}  & 246.20 $\pm$ 16.29 \\
20  & \textbf{97.69 $\pm$ 6.57}   & 126.94 $\pm$ 14.92 & \textbf{174.08 $\pm$ 5.09}  & 174.32 $\pm$ 4.32 \\
40  & \textbf{67.75 $\pm$ 4.11}   & 73.77 $\pm$ 4.19   & \textbf{135.62 $\pm$ 1.26}  & 136.01 $\pm$ 1.91 \\
50  & \textbf{58.10 $\pm$ 3.63}   & 62.00 $\pm$ 4.00   & \textbf{121.30 $\pm$ 1.35}  & 121.50 $\pm$ 0.74 \\
60  & \textbf{51.20 $\pm$ 2.93}   & 56.70 $\pm$ 3.23   & \textbf{114.06 $\pm$ 1.38}  & 114.12 $\pm$ 1.10 \\
120 & \textbf{34.08 $\pm$ 1.62}   & 38.24 $\pm$ 3.40   & \textbf{84.23 $\pm$ 1.10}   & 84.78 $\pm$ 0.67 \\
\hline
\end{tabular}
\end{table}
We have the following observations:
\begin{enumerate}
    \item Results similar to \textbf{MACE} convergence at \textbf{40}.

    \item  Nodal schnet neal consistently outperforms vanilla schnet-neal.
\end{enumerate}
\begin{figure}[htbp]
    \centering
    \begin{minipage}{0.48\linewidth}
        \centering
        \includegraphics[width=\linewidth]{plots/comparsion vanilla schnet vs nodal schnet using only 3rd inetraction and energy.png}
        \caption{Comparison of NEAL with NODAL and Vanilla SchNet results.}
        \label{fig:comparison1}
    \end{minipage}
    \hfill
    \begin{minipage}{0.48\linewidth}
        \centering
        \includegraphics[width=\linewidth]{plots/comparison of gain in neal approach in the case of vanilla and nodal using the 3 r layer interaction enrgies and latents..png}
        \caption{Comparison of NEAL gains with Vanilla and NODAL SchNet (3rd layer interactions only).}
        \label{fig:comparison2}
    \end{minipage}
\end{figure}
\newpage
\subsection{Using only 2nd layer interaction latents and energies}
\begin{table}[h!]
\centering
\begin{tabular}{c|cc|cc}
\hline
\multirow{2}{*}{\textbf{num\_ex}} & \multicolumn{2}{c|}{\textbf{Energy MAE}} & \multicolumn{2}{c}{\textbf{Force MAE}} \\
 & \textbf{Neal} & \textbf{Vanilla} & \textbf{Neal} & \textbf{Vanilla} \\
\hline
5  & \textbf{277.15 $\pm$ 54.76} & 281.67 $\pm$ 55.96 & 321.46 $\pm$ 19.70 & \textbf{321.44 $\pm$ 28.19} \\
10 & \textbf{169.90 $\pm$ 20.80} & 191.67 $\pm$ 9.83 & \textbf{234.44 $\pm$ 5.73} & 246.20 $\pm$ 16.29 \\
20 & \textbf{121.66 $\pm$ 8.26} & 126.97 $\pm$ 14.91 & 177.04 $\pm$ 10.78 & \textbf{174.32 $\pm$ 4.32} \\
40 & \textbf{71.59 $\pm$ 3.19} & 73.78 $\pm$ 4.21 & \textbf{135.39 $\pm$ 2.18} & 136.01 $\pm$ 1.91 \\
50 & 63.24 $\pm$ 5.12 & \textbf{62.00 $\pm$ 3.97} & 122.62 $\pm$ 4.14 & \textbf{121.50 $\pm$ 0.74} \\
60 & \textbf{55.70 $\pm$ 3.64} & 56.70 $\pm$ 3.23 & 114.19 $\pm$ 1.73 & \textbf{114.12 $\pm$ 1.10} \\
\hline
\end{tabular}
\caption{Mean $\pm$ standard deviation of energy and force MAE across when using second layer and second interaction latents.}
\label{tab:energy_force_mae}
\end{table}
\subsection{Using all interaction layers concatenated and Mahalanobis distance for aligning nodal\_energy vector. }
\begin{table}[h!]
\centering
\caption{Mean $\pm$ Std MAE for Energy and Force across different numbers of examples. Lowest per-row values (for Energy and Force separately) are highlighted in bold.}
\renewcommand{\arraystretch}{1.2}
\begin{tabular}{c|cc|cc}
\hline
\multirow{2}{*}{\textbf{Num Examples}} & \multicolumn{2}{c|}{\textbf{Energy MAE}} & \multicolumn{2}{c}{\textbf{Force MAE}} \\
 & \textbf{Neal} & \textbf{Vanilla} & \textbf{Neal} & \textbf{Vanilla} \\
\hline
5   & 283.33 $\pm$ 50.12 & \textbf{281.65 $\pm$ 55.97} & 327.13 $\pm$ 25.12 & \textbf{321.44 $\pm$ 28.19} \\
10  & \textbf{168.92 $\pm$ 18.20} & 191.67 $\pm$ 9.83  & \textbf{233.06 $\pm$ 8.49}  & 246.20 $\pm$ 16.29 \\
20  & 127.29 $\pm$ 13.92 & \textbf{126.96 $\pm$ 14.92} & 176.10 $\pm$ 3.89  & \textbf{174.32 $\pm$ 4.32} \\
40  & 89.09 $\pm$ 13.09  & \textbf{73.77 $\pm$ 4.22}   & \textbf{135.79 $\pm$ 1.91}  & 136.01 $\pm$ 1.91 \\
50  & 74.35 $\pm$ 5.74   & \textbf{61.98 $\pm$ 3.99}   & \textbf{121.16 $\pm$ 2.20}  & 121.50 $\pm$ 0.74 \\
60  & 70.43 $\pm$ 7.53   & \textbf{56.70 $\pm$ 3.22}   & 114.72 $\pm$ 1.58  & \textbf{114.12 $\pm$ 1.10} \\
120 & 45.21 $\pm$ 3.26   & \textbf{38.26 $\pm$ 3.39}   & 86.05 $\pm$ 1.26  & \textbf{84.78 $\pm$ 0.67} \\
\hline
\end{tabular}
\end{table}
This method has poor performance. I am thinking of the following fixes
\begin{enumerate}
    \item Use only the last layer for finding prototypes. Heuristic:- Ideally the last interaction layer has captured all the information so does not need concatenation with previous layers. This makes it much easier to preserve variance via PCA.

    \item Use PCA with a higher number of components.

    \item Use L1 loss in neal instead of L2 loss to help the model adapt more smoothly.
\end{enumerate}
\newpage
\begin{table}[htbp]
\centering
\begin{tabular}{c|cc|cc}
\hline
\textbf{num\_examples} & \multicolumn{2}{c|}{\textbf{Energy MAE}} & \multicolumn{2}{c}{\textbf{Force MAE}} \\
 & \textbf{Neal} & \textbf{Vanilla} & \textbf{Neal} & \textbf{Vanilla} \\
\hline
5   & 283.33 $\pm$ 35.86 & \textbf{281.60 $\pm$ 55.98} & 327.13 $\pm$ 21.17 & \textbf{321.44 $\pm$ 26.80} \\
10  & \textbf{162.81 $\pm$ 22.59} & 190.27 $\pm$ 11.47 & \textbf{244.42 $\pm$ 13.91} & 251.03 $\pm$ 13.41 \\
20  & \textbf{114.31 $\pm$ 6.99} & 123.55 $\pm$ 12.06 & 188.93 $\pm$ 9.77 & \textbf{178.85 $\pm$ 13.23} \\
40  & 86.41 $\pm$ 10.36 & \textbf{73.76 $\pm$ 4.19} & 146.36 $\pm$ 12.90 & \textbf{136.01 $\pm$ 1.91} \\
50  & 72.95 $\pm$ 4.65 & \textbf{62.02 $\pm$ 3.97} & 125.96 $\pm$ 6.85 & \textbf{121.50 $\pm$ 0.74} \\
60  & 65.02 $\pm$ 3.83 & \textbf{56.71 $\pm$ 3.24} & 118.94 $\pm$ 4.99 & \textbf{114.12 $\pm$ 1.10} \\
120 & 42.68 $\pm$ 2.51 & \textbf{38.24 $\pm$ 3.41} & 93.09 $\pm$ 8.03 & \textbf{84.78 $\pm$ 0.67} \\
\hline
\end{tabular}
\caption{Comparison of energy and force MAE (mean $\pm$ std) across different training budgets. A full set of 500 configurations was used to decide the optimal hyper-parameters.}
\label{tab:energy_force_mae}
\end{table}
This shows that model selection criterion is not the reason for degradation in performance.
\subsection{Using all energy latents and only 3rd interaction latents.}
\begin{table}[h!]
\centering
\begin{tabular}{c|cc|cc}
\hline
\textbf{num\_examples} & \multicolumn{2}{c|}{\textbf{Energy MAE}} & \multicolumn{2}{c}{\textbf{Force MAE}} \\
\cline{2-5}
 & \textbf{NeaL} & \textbf{Vanilla} & \textbf{NeaL} & \textbf{Vanilla} \\
\hline
5   & \textbf{257.19 $\pm$ 76.33} & 281.64 $\pm$ 55.96 & \textbf{310.26 $\pm$ 19.96} & 321.44 $\pm$ 28.19 \\
10  & \textbf{143.06 $\pm$ 18.21} & 191.67 $\pm$ 9.83  & \textbf{231.44 $\pm$ 5.52}  & 246.20 $\pm$ 16.29 \\
20  & \textbf{96.88 $\pm$ 7.95}   & 126.96 $\pm$ 14.92 & 178.46 $\pm$ 1.91 & \textbf{174.32 $\pm$ 4.32} \\
40  & \textbf{68.05 $\pm$ 4.67}   & 73.76 $\pm$ 4.23   & 136.47 $\pm$ 1.56 & \textbf{136.01 $\pm$ 1.91} \\
50  & \textbf{60.96 $\pm$ 2.59}   & 61.98 $\pm$ 3.99   & 122.73 $\pm$ 2.96 & \textbf{121.50 $\pm$ 0.74} \\
60  & \textbf{54.31 $\pm$ 3.08}   & 56.71 $\pm$ 3.21   & 115.16 $\pm$ 2.12 & \textbf{114.12 $\pm$ 1.10} \\
120 & \textbf{36.24 $\pm$ 2.79}   & 38.24 $\pm$ 3.40   & 84.97 $\pm$ 1.61  & \textbf{84.78 $\pm$ 0.67} \\
\hline
\end{tabular}
\caption{Mean $\pm$ std for energy and force MAE across different numbers of examples. 
Bold indicates the smaller (better) value for each metric type.}
\label{tab:neal_vanilla_mae}
\end{table}
\newpage
\subsection{Mace with last interaction layers}
We see that both the there is practically no difference in performance.
\begin{table}[h!]
\centering
\begin{tabular}{c p{3.2cm} p{3.2cm} p{3.5cm}}
\hline
\textbf{num\_ex} & \textbf{energy\_mae Last Layer (meV)} & \textbf{energy\_mae Rushikesh (meV)} & \textbf{force\_mae Last Layer (meV/\AA)} \\
\hline
5  & 49.11 $\pm$ 0.25 & \textbf{47.29 $\pm$ 2.61} & 93.84 $\pm$ 2.56 \\
10 & \textbf{31.70 $\pm$ 1.69} & 31.98 $\pm$ 1.28 & 66.33 $\pm$ 3.80 \\
20 & \textbf{22.63 $\pm$ 0.58} & 22.92 $\pm$ 1.86 & 50.32 $\pm$ 0.51 \\
\hline
\end{tabular}
\caption{Comparison of using last interactions vs a concatenation Minimum energy means are highlighted.}
\label{tab:energy_force_mae}
\end{table} 

\section{Using $||F||_2$ based alignment.}
We use the same hyperparameter grid as used for the standard neal approach.
\begin{table*}[h!]
\centering
\begin{tabular}{c|cc|cc}
\hline
\textbf{Num Examples} &
\textbf{NEAL Energy} & \textbf{Vanilla Energy} &
\textbf{NEAL Force} & \textbf{Vanilla Force} \\
\hline

5  &
297.92 $\pm$ 41.17 &
\textbf{281.66 $\pm$ 55.96} &
330.01 $\pm$ 15.13 &
\textbf{321.44 $\pm$ 28.19} \\

10 &
\textbf{166.24 $\pm$ 14.29} &
191.69 $\pm$ 9.81 &
\textbf{231.29 $\pm$ 11.75} &
246.20 $\pm$ 16.29 \\

20 &
\textbf{110.80 $\pm$ 9.70} &
126.96 $\pm$ 14.91 &
\textbf{173.20 $\pm$ 5.59} &
174.32 $\pm$ 4.32 \\

40 &
73.86 $\pm$ 5.57 &
\textbf{73.76 $\pm$ 4.20} &
\textbf{135.16 $\pm$ 1.97} &
136.01 $\pm$ 1.91 \\

50 &
\textbf{60.36 $\pm$ 3.23} &
62.00 $\pm$ 3.98 &
\textbf{120.96 $\pm$ 1.81} &
121.50 $\pm$ 0.74 \\

60 &
\textbf{52.19 $\pm$ 3.00} &
56.71 $\pm$ 3.21 &
\textbf{113.90 $\pm$ 1.66} &
114.12 $\pm$ 1.10 \\

120 &
\textbf{36.56 $\pm$ 1.68} &
38.23 $\pm$ 3.41 &
\textbf{84.39 $\pm$ 1.26} &
84.78 $\pm$ 0.67 \\
\hline
\end{tabular}
\caption{Energy and force MAE (mean $\pm$ std) for force alignment fine tuning vs vanilla finetuning.}
\end{table*}
\\
The improvement in Force MAE is only at budget of 10 examples but energy mae is decreasing as a result of applying this method. 
\\
\textbf{Can we combine the performance increases of both these methods?}

\newpage
\section{Combining force and energy based guidance.}
\begin{table}[h]
\centering
\begin{tabular}{c|cc|cc}
\hline
\textbf{num\_examples} &
\textbf{NEAL Energy} & \textbf{Vanilla Energy} &
\textbf{NEAL Force} & \textbf{Vanilla Force} \\
\hline

5 &
\textbf{280.43 $\pm$ 69.53} & 281.65 $\pm$ 55.98 &
\textbf{312.91 $\pm$ 19.66} & 321.44 $\pm$ 28.19 \\

10 &
\textbf{142.15 $\pm$ 19.82} & 191.69 $\pm$ 9.80 &
\textbf{228.38 $\pm$ 7.67} & 246.20 $\pm$ 16.29 \\

20 &
\textbf{91.18 $\pm$ 11.30} & 126.94 $\pm$ 14.91 &
175.99 $\pm$ 5.40 & \textbf{174.32 $\pm$ 4.32} \\

40 &
\textbf{66.51 $\pm$ 2.45} & 73.77 $\pm$ 4.19 &
\textbf{135.68 $\pm$ 0.96} & 136.01 $\pm$ 1.91 \\
\hline
\end{tabular}
\caption{Energy and force comparison using both energy and force alignment in the ratio 1:1 .}

\end{table}

\end{document}