\documentclass[12pt]{article}

% Essential packages for mathematics
\usepackage{amsmath}    % Advanced math environments
\usepackage{amssymb}    % Additional math symbols
\usepackage{amsthm}     % Theorem environments
\usepackage{mathtools}  % Extensions to amsmath

% Algorithm packages
\usepackage[ruled,vlined]{algorithm2e}  % Popular algorithm package
% Alternative: \usepackage{algorithm,algpseudocode} % For algorithmicx

% Other useful packages
\usepackage{graphicx}   % For including figures
\usepackage{geometry}   % Page layout
\usepackage{hyperref}   % Hyperlinks
\usepackage{cleveref}   % Smart cross-referencing

% Page setup
\geometry{margin=1in}
\setlength{\parindent}{0pt}
\setlength{\parskip}{1em}

% Theorem environments
\newtheorem{theorem}{Theorem}[section]
\newtheorem{lemma}[theorem]{Lemma}
\newtheorem{corollary}[theorem]{Corollary}
\newtheorem{proposition}[theorem]{Proposition}
\theoremstyle{definition}
\newtheorem{definition}[theorem]{Definition}
\newtheorem{example}[theorem]{Example}
\theoremstyle{remark}
\newtheorem{remark}[theorem]{Remark}

% Custom math operators
\DeclareMathOperator{\argmax}{arg\,max}
\DeclareMathOperator{\argmin}{arg\,min}
\DeclareMathOperator{\E}{\mathbb{E}}
\DeclareMathOperator{\Var}{Var}
\DeclareMathOperator{\Cov}{Cov}

% Algorithm2e settings
\SetAlgoNlRelativeSize{0}  % Line numbers same size as text
\SetAlgoNoLine             % Remove vertical lines (comment to keep them)
\DontPrintSemicolon        % Don't print semicolons at end of lines

\title{Paper Plan}
\date{\today}

\begin{document}

\maketitle
\section{Paper Organization}
\textbf{Introduction}
\begin{enumerate}
    \item Importance of material discovery and the role of ML in the field.
    \item The trend of training foundation models on large amounts of simulation data.
    \item Foundation models and the need for fine-tuning to make them usable for unseen material systems.
    \item Need of data-efficient methods due to the high cost of gathering data from simulations unlike other fields like vision with large amount of data present on the web.
\end{enumerate}

\textbf{Prototype based alignment}
\begin{enumerate}
    \item Explain how our method works by clustering similar local neighborhoods together.
    \item Explain why it works from a physics perspective similar neighborhoods should have similar outcomes.
    \item Explain why it works from an ML perspective a bias var tradeoff.
\end{enumerate}
\textbf{Experimental Setup}
\begin{enumerate}
    \item Explain our experimental setup, including k-fold cross validation, hyper-parameter tuning etc.
    \item Prototype Extraction via PCA.
    \item Further ablations with force based guidance and schnet with modified architecture can be discussed.
\end{enumerate}
\textbf{Results}

\section{Datasets}
\textbf{MD22 dataset}
\begin{figure}[htbp]
    \centering
    \includegraphics[width=0.5\linewidth]{plots/md22_dataset.png}
    \caption{MD22 dataset}
    \label{fig:placeholder}
\end{figure}
\end{document}